\section{Coucou !}

This is my abstract!

(\href{https://jupyterbook.org/stable/get-started/export-pdfs/}{https://jupyterbook.org/stable/get-started/export-pdfs/})

I am a book about ... something! Wikipedia has \href{https://en.wikipedia.org/wiki/book}{information about books}: hover over the link for more information.

\begin{itemize}
\item \textbf{jupyter-book init} : initialise a project.
\item \textbf{jupyter book init --write-toc} : autogenerate a Table of Contents
\item \textbf{jupyter-book start} : built-in server that renders your book as a locally-served website.
\item \textbf{jupyter book build --pdf} : building a pdf
\end{itemize}

\subsection{Export PDFs}

\textbf{jupyter book templates list --pdf}

\begin{itemize}
\item 🟢 lapreprint-typst
\item arxiv\_nips : An arXiv compatible template based on the NIPS 2018 Style
\item arxiv\_two\_column : A two column arXiv compatible template
\item frontiers : A template for the Frontiers In range of Journals
\item volcanica : A template for submissions to the Volcanica journal
\item curvenote : A paper styled template using the custom environments and styling to match Curvenote Web
\item plain\_latex : A minimal template that only uses vanilla LaTeX commands and environments
\item plain\_latex\_book : A plain latex book theme
\item springer : Official template for Springer journals
\item ⚠️ ieee-typst : IEEE Template from Typst app, updated as a MyST template
\item ⚠️ plain\_typst\_book : Easily create beautiful books in Typst
\end{itemize}

\textbf{jupyter book templates list --typst lapreprint-typst}

LaPreprint Typst Template : lapreprint-typst
Parts:

\begin{itemize}
\item abstract (required) - An abstract is a short summary of your research paper or report. A good abstract will prepare readers for the detailed information to follow, communicate the essence of the article and help readers take away and remember key points.
\item summary - Plain language summary
\item acknowledgements - Acknowledgements printed in the margin
\item availability - Data availability statement printed in the margin
\end{itemize}

Options:

\begin{itemize}
\item logo (file) - An image path that is shown in the top right of the page
\item kind (string) - The ``kind'' of the content, e.g. ``Original Research'' - shown as the title of the margin content on the first page
-short\_citation (string) - The short citation used in the document header. By default, it is automatically determined from the author names.
\item heading\_numbering (string) - Heading numbering style.
\end{itemize}

Finalement :

\begin{itemize}
\item \textbf{jupyter book build intro.md --tex} et faire pour chaque fichier. Mais impose de nettoyer le tex et d'avoir une classe pour le template avant de compiler en pdf ``à la main''.
\item \textbf{jupyter book build --pdf} compile tous les fichiers en pdf avec le format défini dans le frontmatter de myst.yml :
\end{itemize}

\begin{verbatim}
exports:
    - format: pdf
      template: lapreprint -typst
      articles:
        - intro.md
        - page2.md
        - page3.md
\end{verbatim}